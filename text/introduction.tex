\chapter{Introduction}

\section{Zero Knowledge Proofs of Knowledge}

Today's state-of-the-art cryptosystems are based on the following hard
problems:
\begin{itemize}
\item Discrete Logarithm Problem
\item Factorization Problem
\end{itemize}

Such problems are not solvable using deterministic Turing machines
within polynomial time.  As long as computers are based on
deterministic Turing machines this does not pose as problem.  Simply
scaling the solution space has provided an effective countermeasure so
far. This is all about to change as the world sees the emergence of
quantum computers.

In a post-quantum computer world we would like to seek other problems
which are again not solvable within polynomial time using the machines
of the present.

Another motivation is that in our modern world, privacy is of the
uttermost importance. Let's look at a couple of everyday examples:
\begin{itemize}
\item Buying a public-transport ticket
\item Voting
\item Filling out anonymous questionnaires
\end{itemize}

So far, using traditional methods, privacy was not a big issue here. For
each of the cases:
\begin{itemize}
\item you paid for a piece of paper that nobody (unless he hires
  forensic) could connect to you
\item you circled your vote and put it in a box that nobody (again,
  unless he uses sophisticated methods) could connect to you
\item you filled in your questionnaire that nobody could connect to
  you
\end{itemize}

The real problem comes from trying to put these things into electronic
form. Using smart cards for public transport poses a big problem
w.r.t. privacy as nothing assures the user that the system does not
track him/her. The advantage of putting these things into electronic
form is multiple:
\begin{itemize}
\item less paper is wasted (smaller footprint)
\item less clutter if multiple smart cards are merged into a single
  one
\item less chance of the user losing his smart card if he only has one
\end{itemize}

However, one disadvantage strikes more than the others. The
possibility of the system to track the user and the user being
incapable of protecting his privacy. Traditional authentication
schemes required the user identifying himself so this tracking step
was an inherent property.

The desired property here is not disseminating any knowledge while
still proving to the system that certain properties hold (e.g. the
user has privileges and is allowed to proceed with an action).

\section{HW-SW codesign}

Traditional embedded device programming was separate from hardware
design. The separation went to such extents as having two separate
entities within a company working on each of those fields. The
software part is generic while the hardware part is faster and more
energy efficient.

Today's stringent measures require a bigger inter-dependency. Instead
of searching for a best algorithm tailored for specific hardware or
searching for the best hardware tailored to a specific algorithm/task,
why not design the hardware and software at the same time. This way
bigger decision and trade-offs can be made more granular
\cite{softeninghw}.

\begin{figure}[hbt!]
  \centering
  \begin{tikzpicture}[>=stealth',font=\sffamily,auto,on grid]
    %\input{y_chart_common}
    \coordinate (behaviouralNode) at (135:4cm);
    \coordinate (structuralNode) at (45:4cm);
    \coordinate (physicalNode) at (270:4cm);
    \coordinate (originNode) at (0:0cm);

    \draw[fill] (barycentric cs:behaviouralNode=1.0,originNode=0) circle (2pt);
    \draw[fill] (barycentric cs:behaviouralNode=0.8,originNode=0.2) circle (2pt);
    \draw[fill] (barycentric cs:behaviouralNode=0.6,originNode=0.4) circle (2pt);
    \draw[fill] (barycentric cs:behaviouralNode=0.4,originNode=0.6) circle (2pt);
    \draw[fill] (barycentric cs:behaviouralNode=0.2,originNode=0.8) circle (2pt);

    \draw[fill] (barycentric cs:structuralNode=1.0,originNode=0) circle (2pt);
    \draw[fill] (barycentric cs:structuralNode=0.8,originNode=0.2) circle (2pt);
    \draw[fill] (barycentric cs:structuralNode=0.6,originNode=0.4) circle (2pt);
    \draw[fill] (barycentric cs:structuralNode=0.4,originNode=0.6) circle (2pt);
    \draw[fill] (barycentric cs:structuralNode=0.2,originNode=0.8) circle (2pt);

    \draw[fill] (barycentric cs:physicalNode=1.0,originNode=0) circle (2pt);
    \draw[fill] (barycentric cs:physicalNode=0.8,originNode=0.2) circle (2pt);
    \draw[fill] (barycentric cs:physicalNode=0.6,originNode=0.4) circle (2pt);
    \draw[fill] (barycentric cs:physicalNode=0.4,originNode=0.6) circle (2pt);
    \draw[fill] (barycentric cs:physicalNode=0.2,originNode=0.8) circle (2pt);

    \draw[black!50] (0,0) circle (4.0cm);
    \draw[black!50] (0,0) circle (3.2cm);
    \draw[black!50] (0,0) circle (2.4cm);
    \draw[black!50] (0,0) circle (1.6cm);
    \draw[black!50] (0,0) circle (0.8cm);

    \node [above=1em] at (behaviouralNode) {\textbf{Behavioural Domain}};
    \node [above=1em] at (structuralNode) {\textbf{Structural Domain}};
    \node [below=1em] at (physicalNode) {\textbf{Physical Domain}};

    \draw[-, very thick] (behaviouralNode.south) -- (0,0) node[left,pos=0]{Systems} node[left,pos=0.2]{Algorithms} node[left,pos=0.4]{Register transfers} node[left,pos=0.6]{Logic} node[left,pos=0.8]{Transfer functions};

    \draw[-, very thick] (structuralNode.south) -- (0,0) node[pos=0]{ } node[pos=0.2]{Processors} node[pos=0.4]{ALUs, RAM, etc.} node[pos=0.6]{Gates, flip-flops, etc.} node[pos=0.8]{Transistors};

    \draw[-, very thick] (physicalNode.south) -- (0,0) node[right,pos=0]{Physical partitions} node[right,pos=0.2]{Floorplans} node[right,pos=0.4]{Module layout} node[right,pos=0.6]{Cell layout} node[right,pos=0.8]{Transistor layout};

  \end{tikzpicture}
  \caption{Gajski-Kuhn \index{Gajski-Kuhn Y-chart}Y-chart} 
  \label{figure:gajski_kuhn_y_chart__levels_of_abstraction}
\end{figure}

\filbreak

%%% Local Variables: 
%%% TeX-PDF-mode: t
%%% TeX-master: "thesis"
%%% End: 
