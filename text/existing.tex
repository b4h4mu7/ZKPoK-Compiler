\chapter{Existing Frameworks and Tools}

This chapter starts with an overview of existing frameworks for
implementing Zero Knowledge Proofs of Knowledge. Current examples
include CACE Project Zero Knowledge Compiler, ZKPDL and IBM
Idemix. CACE project will be covered in more detail while ZKPDL will
be briefly mentioned. The chapter then continues on describing other
tools that will be used for this thesis: ANTLR, a parser generator
tool producing LL(*) parsers and LLVM, a compiler infrastructure
framework that has seen wide use recently in many fields relating to
compilers and computers in general. ANTLR and LLVM will be used to
make a custom compiler so some knowledge about them is necessary. Next
is GEZEL, a cycle-true cosimulation environment which can also
generate VHDL or Verilog.

\section{CACE Project}

\subsection{Framework Overview}

The CACE (Computer Aided Cryptography Engineering)
Project\footnote{http://www.cace-project.eu} was an European project
aiming at developing a toolbox for security software. It attempted to
ease the creation of cryptographic software for those outside the
domain. The goals were:
\begin{itemize}
\item Automatic translation from natural specification - the
  term natural is taken from the user's perspective, meaning something
  ``natural'' for the user, not dwelling too much into the specific
  niches of cryptography, giving an abstract overview
\item Automatic security awareness, analysis and corrections - to be
  able to detect side channels that are unintentionally introduced,
  warn the user and offer corrective actions
\item Automatic optimization for diverse platforms - different
  platforms are suited for different operations, assume different
  usage patterns etc\ldots, the toolbox should be as most insensitive
  as it can to the platform it is implemented on
\end{itemize}

Apart from these goals that deal with the end-user, the project had
strategic goals of opening a new field of research and promoting
automatic tools when it comes to crypto software. The project itself
was split into multiple working groups:
\begin{itemize}
\item WP1 Automating Cryptographic Implementation - dealing with the
  low level crypto operations, searching and identifying side channel
  attacks, providing a domain specific language
\item WP2 Accelerating Secure Network - dealing with basic operations
  for module intercommunication
\item WP3 Bringing Proofs of Knowledge to Practice - dealing with
  implementing a compiler for Proofs of Knowledge
\item WP4 Securing Distributed Management of Information - dealing
  with higher operations for module intercommunication
\item WP5 Formal Verification and Validation - dealing with analysis
  of the correctness, assuring the user of the protocol validity
\end{itemize}

The WP3 working group is of the importance for this thesis as it deals
directly with proofs of knowledge. The end result of the working group
was a compiler along with a specification language (PSL) and an
intermediate language (PIL). The compiler has the following typical
flow (as depicted in Figure \ref{fig:cace_workflow}):
\begin{enumerate}
\item Write PSL (Protocol Specification Language) - the user specifies the
  protocol using Camenisch-Stadler notation (see Sub-subsection
  \ref{subsubsec:camenisch_stadler})
\item Generate PIL (Protocol Interface Language) from PSL - the
  framework transform it into a lower-level language for specifying
  operations
\item Generate C or Java code from PIL - the framework generates a
  code that is possible to compile and execute on a target
  architecture
\item (Optional) Verify PIL code using PVT - the framework applies the
  formal verification to the PIL code
\item (Optional) Generate LaTeX from PIL - the framework generates
  \LaTeX code with the protocol flow
\end{enumerate}

\begin{figure}[hb!]
  \centering
  \begin{tikzpicture}[>=stealth,level distance=1.5cm,font=\tiny]
    \tikzstyle{edge from parent}=[draw,->]

    \Tree [.\node[language](psl){Protocol \\ Specification \\ Language (PSL)};
      [.\node[compiler](pc){Protocol \\ Compiler};
        [.\node[language](pil){Protocol \\ Implementation \\ Language (PIL)};
          [.\node[compiler](c){C}; \node[language](code){Code};]
          [.\node[compiler](latex){\LaTeX}; \node[language](doc){Documentation};]
        ]
      ]
    ]

    \node[compiler] (pvt)         [right=of pc,anchor=west]          {Protocol \\ Verification \\ Toolbox}
    child {node[language] {Proof of \\ Soundness}};

    \node[compiler] (sigma) [left=of pil.north west,anchor=center] {$\Sigma 2 N I Z K$};
    \node[compiler] (cost) [left=of pil.south west,anchor=center] {Costs};

    \draw[<->] (sigma) -- (pil);
    \draw[<->] (cost) -- (pil);

    \draw[->] (psl) -- (pvt);
    \draw[->] (pil) -- (pvt);
  \end{tikzpicture}
  \caption{CACE Project Zero Knowledge Compiler typical workflow \cite{CACE}}
  \label{fig:cace_workflow}
\end{figure}

\subsection{PSL}
\label{subsec:psl}

The Protocol Specification Language (PSL) is a high level language of
CACE Project WP3 for specifying Proofs of Knowledge based on
Camenisch-Stadler notation. It allows specification of complex
$\Sigma$ protocols \cite{CACE, yaczk}. The explanation is best given
following an example of a simple protocol, Schnorr's Identification
Protocol (see Sub-subsection \ref{subsubsec:schnorr_protocol}). The
Camenisch-Stadler notation is:
\[
  \textrm{ZPK}\left[ (x): y = g^x \right]
\]

The PSL language is structured into blocks, and for the Schnorr
example the blocks are as follows:
\begin{itemize}
\item Declarations - specifying all the variables used within
  the protocol
\begin{lstlisting}[language=PSL]
Declarations {
  Prime(1024) p;
  Prime(160) q;
  Zmod+(q) x;
  Zmod*(p) g, y;
}
\end{lstlisting}
  This example shows how to declare prime numbers as well as elements
  of a residue group. Here, $p$ is declared as a prime of $1024$ bits
  and $q$ is declared as a prime of $160$ bits, $x \in Z_q^+$ and $g,
  y \in Z_p^*$.
\item Input - specifying which of the variables are public and which
  are private (to the Verifier or the Prover)
\begin{lstlisting}[language=PSL]
Inputs {
  Public := y,p,q,g;
  ProverPrivate := x;
}
\end{lstlisting}
  This example shows how to specify which are public known variables
  and which are known only to the prover. It is also possible to
  specify a variable known only to the verifier.
\item Properties - specifying the properties of the protocol
\begin{lstlisting}[language=PSL]
Properties {
  KnowledgeError := 80;
  SZKParameter := 80;
  ProtocolComposition := P_1;
}
\end{lstlisting}
  This example shows how to specify the knowledge error, which is
  $2^{-80}$ in this case. The tightness is specified at $2^{-80}$ as
  well. The protocol composition allows to specify multiple protocols
  via AND, OR, XOR. Since this is a simple case, only one protocol is
  used.

\item Specifying the protocols itself (the homomorphism to
  use and the relation to be proven)
\begin{lstlisting}[language=PSL]
SigmaPhi P_1 {
  Homomorphism (phi : Zmod+(q) -> Zmod*(p) : (a) |-> (g^a));
  ChallengeLength := 80;
  Relation ((y) = phi(x));
}
\end{lstlisting}
  The relation to be proven is $y = g^x$ which is specified as the
  homomorphism $\phi(a) : Z_q^+ \rightarrow Z_p^* := g^a$. The
  ChallengeLength allows to customize the protocol for certain
  devices.  The generated protocol will be repeated until the required
  knowledge error is met. For example, a challenge length of $1$ will
  repeat the protocol $80$ times to satisfy the knowledge error of
  $2^{-80}$.

\end{itemize}

The previous blocks combined give a complete PSL specification of
Schnorr's Identification Protocol:
\lstinputlisting[language=PSL]{example.psl}

\subsection{PIL}
\label{subsec:pil}

The low level/intermediate language of the CACE Project WP3 is PIL
(Protocol Implementation Language). The language itself gives all the
details of a protocol and is meant to be easy to understand and
learn. This can aid in verifying the correctness from a user's point of
view. The constructs of the language are completely specified and
allow an automated verification.

As a language on its own, PIL has support for the following features:
\begin{itemize}
\item global shared constants (parameters)
\begin{lstlisting}[language=PIL]
Common (
Z l_e = 1024;
Z SZKParameter = 80;
Prime(1024) n
) {
...
}
\end{lstlisting}

\item global constants (parameters)
\begin{lstlisting}[language=PIL]
Prover (
Zmod+(q) x;
Zmod+(q) v
) {
...
}
\end{lstlisting}

\item global variables
\begin{lstlisting}[language=PIL]
Prover (
...
) {
Zmod+(q) s, r;
...
}
\end{lstlisting}

\item conditionals
\begin{lstlisting}[language=PIL]
IfKnown(...) {

} Else {

}
\end{lstlisting}

\item loops
\begin{lstlisting}[language=PIL]
For i In [1,2] {
  ...
}
\end{lstlisting}

\item functions
\begin{lstlisting}[language=PIL]
Def (Zmod*(p) _t_1): Round1(Void) {
 _r_1 := Random(Zmod+(q));
 _t_1 := (g^_r_1);
}
\end{lstlisting}

\item predicates
\begin{lstlisting}
x := Random(Zmod+(q));
CheckMembership(x, Zmod+(q));
Verify(x == x);
\end{lstlisting}

\item type alias
\begin{lstlisting}
_C = Int(80) _c;
\end{lstlisting}

\end{itemize}

Proof entities are specified as blocks and there is always a Common
block with all declarations and definitions visible to all other
blocks. Each block can define multiple functions that have inputs and outputs
defined. A function consists of assignments or loops or conditional
flow. The execution order is specified via block function pairs:
\begin{lstlisting}[language=PIL]
ExecutionOrder := (Prover.Round0, Verifier.Round0, Prover.Round1, Verifier.Round1, Prover.Round2, Verifier.Round2);
\end{lstlisting}
The communication itself is specified via these functions. The inputs
of the current function must match the output of the previous
function. For example, the outputs of Round0 from Prover must match
the inputs of Round0 from Verifier.

Again, the Schnorr protocol is used as an example, automatically
generated from the PSL that was given in Sub-section \ref{subsec:psl}.
\lstinputlisting[language=PIL]{example.pil}

\section{Related Frameworks}

\subsection{ZKPDL/Cashlib}

ZKPDL is a description language for describing Zero Knowledge Proofs
of Knowledge. It is used by the framework Cashlib to implement
e-cash. The framework uses an interpreter based approach and applies
result caching to speed up computations. Unlike PIL, ZKDPL is not
Turing complete as it does not allow branching or conditionals. Also,
ZKPDL only supports non-interactive proofs of knowledge
\cite{zkpdl}. ZKPDL does allow generation of parameters which PIL
lacks \cite{yaczk} but this can be mitigated by defining and using
constant expressions.

Because of its generality, a standard notation of specifying proofs of
knowledge as well as the ability to formally verify code and generate
documentation, the CACE Project Zero Knowledge Compiler will be the
framework of choice for extension in this thesis.

\section{Other Tools}

\subsection{ANTLR}

ANTLR\footnote{http://www.antlr.org} (ANother Tool For Language
Recognition) is a parser generator tool that generates LL(*)
parsers. The tool accepts grammar definitions as input files and
produces output in the target language which can be chosen among C,
Java or Python.

The input to ANTLR is a context-free grammar that must be of the LL
form. This means that there should be no left recursion or ambiguities
when encountering the first element on the left. Left factoring is
usually used to solve this, but this can sometimes lead to rules which
have counter intuitive representation. The grammar can be augmented with
syntactic and semantic predicates to cope with this \cite{ANTLR,ANTLR2}.

ANTLR can generate both lexers and parser. The two grammars can be
combined in a single file.  In such cases, parser rules are written in
lowercase, while the lexer rules are written in uppercase. Upon
generation, two separate entities will be created, a parser source and
a lexer source in the target language (as illustrated in Figure
\ref{fig:antlr_parser_lexer}).

\begin{figure}[hb!]
  \centering
  \begin{tikzpicture}[>=stealth, font=\tiny]
    \tikzstyle{edge from parent}=[draw,->]

    \Tree[.\node[language](parser_g){Grammar \\ *.g};
      [.\node[compiler](antlr){ANTLR};
        [.\node[compiler](lexer){Lexer};]
        [.\node[compiler](parser){Parser};]
      ]
    ]
  \end{tikzpicture}
  \caption{ANTLR Parser/Lexer generation}
  \label{fig:antlr_parser_lexer}
\end{figure}

The output of the parser generated by ANTLR is a Parse Tree. ANTLR
also allows specifying a tree transformation to apply to this Parse
Tree. This can be used to automatically generate an Abstract Syntax
Tree (AST).

ANTLR can also generate a tree parser/walker that visit each node of
the AST and applies a certain operation or produces a certain
output. The generation of such a walker is illustrated in Figure
\ref{fig:antlr_tree_walker}.

\begin{figure}[hb!]
  \centering
  \begin{tikzpicture}[>=stealth, font=\tiny]
    \tikzstyle{edge from parent}=[draw,->]

    \Tree[.\node[language](parser_g){Tree Grammar \\ *.g};
      [.\node[compiler](antlr){ANTLR};
        [.\node[compiler](lexer){Tree walker};]
      ]
    ]
  \end{tikzpicture}
  \caption{ANTLR Tree walker generation}
  \label{fig:antlr_tree_walker}
\end{figure}

\subsection{LLVM}

LLVM\footnote{http://llvm.org} is a compiler framework designed to
support transparent, life-long program analysis and transformation for
arbitrary programs, by providing high-level information to compiler
transformations at compile-time, link-time, run-time, and in idle time
between runs \cite{LLVM:CGO04}.

Traditional compilers were tailored for only a few languages (with the
exception of GCC). However, all traditional compilers suffer from the
large inter-dependency of the basic blocks (Front-end, Optimizer,
Back-end). LLVM tries to solve this by providing an intermediate form
called the LLVM IR. A typical flow involving the basic blocks is
depicted in Figure \ref{fig:llvm_flow}.

\begin{figure}[hb!]
  \centering
   \begin{tikzpicture}[>=stealth]
    \tikzstyle{lang}=[rectangle,draw=black,thin,font=\tiny,inner
    sep=0pt, align=center,minimum width=2.7cm,minimum height=2.2em]

    \tikzstyle{txt}=[font=\tiny]

    \node[lang](llvm_opt){LLVM \\ Optimizer};
    \node[lang](ppc_back)[right=1 cm of llvm_opt]{LLVM \\ PowerPC Backend};
    \node[txt](ppc)[right=of ppc_back]{PowerPC};
    \node[lang](x86_back)[above of=ppc_back]{LLVM \\ x86 Backend};
    \node[txt](x86)[right=of x86_back]{x86};
    \node[lang](arm_back)[below of=ppc_back]{LLVM \\ ARM Backend};
    \node[txt](arm)[right=of arm_back]{ARM};

    \node[lang](gcc_front)[left=1 cm of llvm_opt]{llvm-gcc \\ Frontend};
    \node[txt](fortran)[left=of gcc_front]{Fortran};
    \node[lang](clang_front)[above of=gcc_front]{Clang C/C++/ObjC \\ Frontend};
    \node[txt](c)[left=of clang_front]{C};
    \node[lang](ghc_front)[below of=gcc_front]{GHC \\ Frontend};
    \node[txt](haskell)[left=of ghc_front]{Haskell};

    \draw[->] (clang_front.east) -- (llvm_opt.west);
    \draw[->] (gcc_front.east) -- (llvm_opt.west);
    \draw[->] (ghc_front.east) -- (llvm_opt.west);

    \draw[->] (llvm_opt.east) -- (x86_back.west);
    \draw[->] (llvm_opt.east) -- (ppc_back.west);
    \draw[->] (llvm_opt.east) -- (arm_back.west);

    \draw[->] (x86_back) -- (x86);
    \draw[->] (ppc_back) -- (ppc);
    \draw[->] (arm_back) -- (arm);

    \draw[->] (c) -- (clang_front);
    \draw[->] (fortran) -- (gcc_front);
    \draw[->] (haskell) -- (ghc_front);
  \end{tikzpicture}
  \caption{LLVM typical workflow \cite{llvm_general}}
  \label{fig:llvm_flow}
\end{figure}

Due to its modularity, LLVM has recently seen increased usage in a number
of independent fields:
\begin{itemize}
\item implementing a C/C++ compiler (Clang)
\item implementing C-to-HDL translation
\item implementing a Haskell compiler (GHC)
\item implementing Secure Virtual Architectures (SVA)
\item implementing dynamic translation
\item implementing OpenGL drivers (Mac OS X)
\item implementing OpenCL drivers (AMD)
\end{itemize}

\subsubsection{LLVM IR}

The LLVM IR is the intermediate representation language of the LLVM
project. On its own it is a first-class language with well defined
semantics \cite{llvm_general, llvm_master_thesis}. Variables are in the SSA (Static Single
Assignment) form meaning that they can be only assigned once and they
keep that value for their entire lifetime. All the values residing in
memory need to be loaded to a variable first and stored back to memory
if they wish to be saved. Instructions operate solely upon
variables. In this respect, the LLVM IR resembles the assembly
language of an infinitely many registers Load-Store based RISC
processor.

\begin{lstlisting}[language=C]
unsigned add1(unsigned a, unsigned b) {
  return a+b;
}

// Perhaps not the most efficient way to add two numbers.
unsigned add2(unsigned a, unsigned b) {
  if (a == 0) return b;
  return add2(a-1, b+1);
}
\end{lstlisting}

\begin{lstlisting}
define i32 @add1(i32 %a, i32 %b) {
entry:
  %tmp1 = add i32 %a, %b
  ret i32 %tmp1
}

define i32 @add2(i32 %a, i32 %b) {
entry:
  %tmp1 = icmp eq i32 %a, 0
  br i1 %tmp1, label %done, label %recurse

recurse:
  %tmp2 = sub i32 %a, 1
  %tmp3 = add i32 %b, 1
  %tmp4 = call i32 @add2(i32 %tmp2, i32 %tmp3)
  ret i32 %tmp4

done:
  ret i32 %b
}
\end{lstlisting}

The central concept in constructing LLVM IR is the Module. Each module
consists of functions, global variables and symbol table entries.
Modules can be combined using the LLVM linker \cite{llvm_ir}.

\subsection{GEZEL}

GEZEL is a cycle-accurate hardware description language (HDL) using the
Finite-State-Machine + Datapath (FSMD) model \cite{gezel}.

The basic element is a Signal Flow Graph. It groups operations that
are to be executed concurrently in the same clock cycle.
\begin{lstlisting}[language=GEZEL]
sfg increment {
  a = a + 1;
}
\end{lstlisting}
One or more of these SFGs are used to form a datapath which is the
main building block. It is the smallest GEZEL unit that can stand on
its own and be simulated \cite{gezel}. A datapath can be thought of as
a \emph{module} in Verilog or an \emph{entity} in VHDL. Here is a
full contained example of a counter in GEZEL:
\begin{lstlisting}[language=GEZEL]
dp counter(out value : ns(2)) {
   reg c : ns(2);
   always {
     value = c;
     c = c + 1;
     $display("Cycle ", $cycle, ": counter = ", value);
   }
}

system S {
  counter;
}
\end{lstlisting}

Figure \ref{fig:gezel_workflow} shows how the GEZEL language
can be used as an input to:
\begin{itemize}
\item fdlvhd - a generator that can generate synthesizeable VHDL or
  Verilog
\item fdlsim - a cycle accurate simulator used to verify and validate
  the design
\item gplatform - a co-simulation tool used for HW/SW co-design purposes
\end{itemize}

\begin{figure}[hb!]
  \centering
  \begin{tikzpicture}[>=stealth, font=\tiny]
    \tikzstyle{edge from parent}=[draw,->]

    \Tree[.\node[language](fdl){GEZEL language \\ *.fdl};
      [.\node[compiler](fdlvhd){Code Generator \\ fdlvhd};
        [.\node[language](vhdl){VHDL \\ Verilog};]
      ]
      [.\node[compiler](fdlsim){Simulator \\ fdlsim};
        [.\node[language](sim){Verification \\ Profiling};]
      ]
      [.\node[compiler](gplatform){CoSimualtor \\ gplatform};
        [.\node[language](cosim){Verification \\ Profiling};]
      ]
    ]
  \end{tikzpicture}
  \caption{GEZEL workflow \cite{gezel}}
  \label{fig:gezel_workflow}
\end{figure}

The co-simulation tool allows to cosimulate GEZEL designs with
instruction-set simulations \cite{gezel}. Supported processors are
ARM, AVR, 8051, MicroBlaze and PicoBlaze. The cosimulation tool allows
for designing a processor-coprocessor pair for a general purpose
processor and a custom dedicated coprocessor.

%%% Local Variables:
%%% TeX-PDF-mode: t
%%% TeX-master: "thesis"
%%% End:
