\chapter{Existing Frameworks and Tools}

The goal of this chapter is to review existing frameworks for
implementing Zero Knowledge Proofs of Knowledge as well as tools that
will be used to design our custom framework.

The chapter starts by analyzing CACE Project Zero Knowledge Compiler,
which will be the basis of our custom framework. A brief overview of
the framework is given followed by the Protocol Specification Language
(PSL) and the Protocol Implementation Language (PIL) overview. More
attention is given to PIL since it will be the language of choice of
our custom framework.

A brief description of related frameworks follows, namely ZKPDL and
ZKCrypt. Only a short overview is given and a high-level comparison
with our custom framework. The chapter then continues on describing
tools that will be used to build our custom framework:
\begin{itemize}
\item ANTLR is a parser generator tool producing LL(*) parsers, we
  will use it to construct a PIL parser which will be the front-end of
  our framework.
\item LLVM is a compiler infrastructure framework that has recently
  seen wide use in many fields relating to computer science. We will
  use it as a middle layer along with its intermediate language LLVM
  IR.
\item GMP is a popular multi-precision library used within the CACE
  ZKC. The changes that will be introduced to PIL later on will make
  it incompatible with the CACE ZKC, therefore we need to provide this
  implementation as well.
\item GEZEL is a cycle-true cosimulation environment and a hardware
  description language. We will target GEZEL language in our
  framework, allowing us to perform HW-SW co-design.
\end{itemize}

\section{CACE Project Zero Knowledge Compiler}
\label{sec:cace}

\subsection{Framework Overview}

The CACE (Computer Aided Cryptography Engineering)
Project\footnote{http://www.cace-project.eu} was an European project
aiming at developing a toolbox for security software. It attempted to
ease the creation of cryptographic software for those outside the
domain by setting the following goals:
\begin{itemize}
\item Automatic translation from natural specification - the
  term natural is taken from the user's perspective, meaning something
  ``natural'' for the user, not dwelling too much into the specific
  niches of cryptography, giving an abstract overview
\item Automatic security awareness, analysis and corrections - to be
  able to detect side channels that are unintentionally introduced,
  warn the user and offer corrective actions
\item Automatic optimization for diverse platforms - different
  platforms are suited for different operations, assume different
  usage patterns etc\ldots, the toolbox should be as most insensitive
  as it can to the platform it is implemented on
\end{itemize}

Apart from these goals that deal with the end-user, the project had
strategic goals of opening a new field of research and promoting
automatic tools when it came to cryptography software. The project
itself was split into multiple working groups:
\begin{itemize}
\item WP1 Automating Cryptographic Implementation - dealing with the
  low level crypto operations, searching and identifying side channel
  attacks, providing a domain specific language
\item WP2 Accelerating Secure Network - dealing with basic operations
  for module intercommunication
\item WP3 Bringing Proofs of Knowledge to Practice - dealing with
  implementing a compiler for Proofs of Knowledge
\item WP4 Securing Distributed Management of Information - dealing
  with higher operations for module intercommunication
\item WP5 Formal Verification and Validation - dealing with analysis
  of the correctness, assuring the user of the protocol validity
\end{itemize}

The WP3 working group is of the importance for this thesis as it deals
directly with proofs of knowledge. The end result of the working group
was a compiler along with a specification language (PSL) and an
intermediate language (PIL). Figure \ref{fig:cace_workflow} depicts the
typical flow of the CACE ZKPK Compiler:
\begin{enumerate}
\item The ZKPK to be implemented is specified in Protocol
  Specification Language (PSL). PSL is based on Camenisch-Stadler
  notation (see Sub-subsection \ref{subsubsec:camenisch_stadler}) and
  allows the specification of complex $\Sigma$ protocols in an easy way.

\item The Protocol Compiler transforms the specification in PSL into
  an implementation in Protocol Implementation Language (PIL). PIL is
  a Turing-complete language on its own and it also allows for
  automated verification.

\item The framework transforms PIL code into an implementation in C or
  Java that can be compiled and executed on a target architecture. Two
  modules are generated along with an additional support library
  providing math and communication. The two modules are a prover and a
  verifier, using network sockets for communication.

\item The PIL code is verified using the Protocol Verification Toolbox
  (PVT). On input the specification in PSL and the implementation in
  PIL, PVT employs the theorem prover Isabelle/HOL \cite{isabelle_hol}
  to create a proof of the soundness guarantees of the generated ZKPK
  implementation.

\item  Human readable documentation in \LaTeX{} is also generated from PIL code.

\end{enumerate}

\begin{figure}[hb!]
  \centering
  \begin{tikzpicture}[>=stealth,level distance=1.5cm,font=\tiny]
    \tikzstyle{edge from parent}=[draw,->]

    \Tree [.\node[language](psl){Protocol \\ Specification \\ Language (PSL)};
      [.\node[compiler](pc){Protocol \\ Compiler};
        [.\node[language](pil){Protocol \\ Implementation \\ Language (PIL)};
          [.\node[compiler](c){C}; \node[language](code){Code};]
          [.\node[compiler](latex){\LaTeX}; \node[language](doc){Documentation};]
        ]
      ]
    ]

    \node[compiler] (pvt)         [right=of pc,anchor=west]          {Protocol \\ Verification \\ Toolbox}
    child {node[language] {Proof of \\ Soundness}};

    \node[compiler] (sigma) [left=of pil.north west,anchor=center] {$\Sigma 2 N I Z K$};
    \node[compiler] (cost) [left=of pil.south west,anchor=center] {Costs};

    \draw[<->] (sigma) -- (pil);
    \draw[<->] (cost) -- (pil);

    \draw[->] (psl) -- (pvt);
    \draw[->] (pil) -- (pvt);
  \end{tikzpicture}
  \caption{CACE Project Zero Knowledge Compiler typical workflow \cite{CACE}}
  \label{fig:cace_workflow}
\end{figure}

\subsection{PSL}
\label{subsec:psl}

The Protocol Specification Language (PSL) is a high level language of
CACE Project WP3 for specifying Proofs of Knowledge based on
Camenisch-Stadler notation, allowing specification of complex $\Sigma$
protocols \cite{CACE, yaczk}. PSL is best explained following an
example of a simple protocol, Schnorr's Identification Protocol (see
Sub-subsection \ref{subsubsec:schnorr_protocol}). We start from the
Camenisch-Stadler notation for Schnorr's Identification Protocol:
\[
  \textrm{ZPK}\left[ (x): y = g^x \right]
\]
The PSL language is structured into blocks, and for the Schnorr
example the blocks are as follows:
\begin{itemize}
\item Declarations - specifying all the variables used within
  the protocol
\begin{lstlisting}[language=PSL]
Declarations {
  Prime(1024) p;
  Prime(160) q;
  Zmod+(q) x;
  Zmod*(p) g, y;
}
\end{lstlisting}
  This example shows how to declare prime numbers as well as elements
  of a residue group. Here, $p$ is declared as a prime of $1024$ bits
  and $q$ is declared as a prime of $160$ bits, $x \in Z_q^+$ and $g,
  y \in Z_p^*$.
\item Input - specifying which of the variables are public and which
  are private (to the Verifier or the Prover)
\begin{lstlisting}[language=PSL]
Inputs {
  Public := y,p,q,g;
  ProverPrivate := x;
}
\end{lstlisting}
  This example shows how to specify which are public known variables
  and which are known only to the prover. It is also possible to
  specify a variable known only to the verifier.
\item Properties - specifying the properties of the protocol
\begin{lstlisting}[language=PSL]
Properties {
  KnowledgeError := 80;
  SZKParameter := 80;
  ProtocolComposition := P_1;
}
\end{lstlisting}
  This example shows how to specify the knowledge error, which is
  $2^{-80}$ in this case. The tightness is specified at $2^{-80}$ as
  well. The protocol composition allows to specify multiple protocols
  via AND, OR and XOR. Since this is a simple case, only one protocol is
  used.

\item Specifying the protocols itself (the homomorphism to
  use and the relation to be proven)
\begin{lstlisting}[language=PSL]
SigmaPhi P_1 {
  Homomorphism (phi : Zmod+(q) -> Zmod*(p) : (a) |-> (g^a));
  ChallengeLength := 80;
  Relation ((y) = phi(x));
}
\end{lstlisting}
  The relation to be proven is $y = g^x$ which is specified as the
  homomorphism $\phi(a) : Z_q^+ \rightarrow Z_p^* := g^a$. The
  ChallengeLength allows to customize the protocol for certain
  devices.  The generated protocol will be repeated until the required
  knowledge error is met. For example, a challenge length of $1$ will
  repeat the protocol $80$ times to satisfy the knowledge error of
  $2^{-80}$.

\end{itemize}

The previous blocks combined give a complete PSL specification of
Schnorr's Identification Protocol:
\lstinputlisting[language=PSL]{example.psl}

\subsection{PIL}
\label{subsec:pil}

The Protocol Implementation Language (PIL) is the low
level/intermediate language of CACE ZKPK Compiler. The language itself
gives all the rounds and computations of a protocol and is meant to be
easy to understand and learn, aiding verification of correctness from
a user's point of view. PIL is also Turing-complete with completely
specified semantics allowing automatic verification. The following
features are supported in PIL:
\begin{itemize}
\item global shared constants (parameters)
\begin{lstlisting}[language=PIL]
Common (
Z l_e = 1024;
Z SZKParameter = 80;
Prime(1024) n
) {
...
}
\end{lstlisting}

\item global constants (parameters)
\begin{lstlisting}[language=PIL]
Prover (
Zmod+(q) x;
Zmod+(q) v
) {
...
}
\end{lstlisting}

\item global variables
\begin{lstlisting}[language=PIL]
Prover (
...
) {
Zmod+(q) s, r;
...
}
\end{lstlisting}

\item conditionals
\begin{lstlisting}[language=PIL]
IfKnown(...) {

} Else {

}
\end{lstlisting}

\item loops
\begin{lstlisting}[language=PIL]
For i In [1,2] {
  ...
}
\end{lstlisting}

\item functions
\begin{lstlisting}[language=PIL]
Def (Zmod*(p) _t_1): Round1(Void) {
 _r_1 := Random(Zmod+(q));
 _t_1 := (g^_r_1);
}
\end{lstlisting}

\item predicates
\begin{lstlisting}
x := Random(Zmod+(q));
CheckMembership(x, Zmod+(q));
Verify(x == x);
\end{lstlisting}

\item type alias
\begin{lstlisting}
_C = Int(80) _c;
\end{lstlisting}

\end{itemize}

Proof entities are specified as blocks and there is always a Common
block with declarations and definitions visible to all other
blocks. Each block can define multiple functions, each having inputs
and outputs and consisting of assignments, loops or conditional
flow. The execution order is specified via block function pairs:
\begin{lstlisting}[language=PIL]
ExecutionOrder := (Prover.Round0, Verifier.Round0, Prover.Round1, Verifier.Round1, Prover.Round2, Verifier.Round2);
\end{lstlisting}
The communication itself is specified via these functions with inputs
of the current function matching the output of the previous
function. For example, the outputs of Round0 from Prover match the
inputs of Round0 from Verifier.

Again, the Schnorr protocol is used as an example, automatically
generated from the PSL that was given in Sub-section \ref{subsec:psl}.
\lstinputlisting[language=PIL]{example.pil}

\section{Related Frameworks}

Although the CACE Project Zero Knowledge Compiler (CACE ZKC) will be
the framework of choice for extension in this thesis, ZKPDL and
ZKCrypt will be presented as well for the sake of completeness. It is
mostly a brief overview along with a brief comparison with CACE ZKC.

\subsection{ZKPDL/Cashlib}

ZKPDL is a ZKPK description language used by the Cashlib framework to
implement e-cash. The framework uses an interpreter based approach and
applies result caching to speed up computations. Unlike PIL, ZKDPL is
not Turing-complete and only supports non-interactive proofs of
knowledge \cite{zkpdl}. ZKPDL does allow generation of parameters
which PIL lacks \cite{yaczk} but this can be mitigated by defining and
using constant expressions.

\subsection{ZKCrypt}
\label{subsec:zkcrypt}

ZKCrypt is a framework built atop the CACE ZKC framework
\cite{zkcrypt}. It leverages
CertiCrypt\footnote{http://www.msr-inria.inria.fr/projects/sec/certicrypt/index.html}
to produce a formal assurance of the correctness. Where CACE only
allows proof of soundness using Isabelle/HOL, ZKCrypt allows formal
assurance of the implementation satisfying completeness, proof of
knowledge and zero knowledge properties.

ZKCrypt generates formal evidence of each of the CACE ZKC compilation
steps except code generation and, as such, is complementary to the
framework developed within the course of this thesis.

\section{Other Tools}

\subsection{ANTLR}

ANTLR\footnote{http://www.antlr.org} (ANother Tool For Language
Recognition) is a parser generator tool that generates LL(*)
parsers. The tool accepts grammar definitions as input files and
produces output in the target language which can be chosen among C,
Java or Python.

The input to ANTLR is a context-free grammar that must be of the LL
form. This means that there should be no left recursion or ambiguities
when encountering the first element on the left. Left factoring is
usually used to solve this, but this can sometimes lead to rules which
have counter intuitive representation. The grammar can be augmented with
syntactic and semantic predicates to cope with this \cite{ANTLR,ANTLR2}.

ANTLR can generate both lexers and parser. The two grammars can be
combined in a single file.  In such cases, parser rules are written in
lowercase, while the lexer rules are written in uppercase. Upon
generation, two separate entities will be created, a parser source and
a lexer source in the target language (as illustrated in Figure
\ref{fig:antlr_parser_lexer}).

\begin{figure}[hb!]
  \centering
  \begin{tikzpicture}[>=stealth, font=\tiny]
    \tikzstyle{edge from parent}=[draw,->]

    \Tree[.\node[language](parser_g){Grammar \\ *.g};
      [.\node[compiler](antlr){ANTLR};
        [.\node[compiler](lexer){Lexer};]
        [.\node[compiler](parser){Parser};]
      ]
    ]
  \end{tikzpicture}
  \caption{ANTLR Parser/Lexer generation}
  \label{fig:antlr_parser_lexer}
\end{figure}

The output of the parser generated by ANTLR is a Parse Tree. ANTLR
also allows specifying a tree transformation to apply to this Parse
Tree. This can be used to automatically generate an Abstract Syntax
Tree (AST).

ANTLR can also generate a tree parser/walker that visit each node of
the AST and applies a certain operation or produces a certain
output. The generation of such a walker is illustrated in Figure
\ref{fig:antlr_tree_walker}.

\begin{figure}[hb!]
  \centering
  \begin{tikzpicture}[>=stealth, font=\tiny]
    \tikzstyle{edge from parent}=[draw,->]

    \Tree[.\node[language](parser_g){Tree Grammar \\ *.g};
      [.\node[compiler](antlr){ANTLR};
        [.\node[compiler](lexer){Tree walker};]
      ]
    ]
  \end{tikzpicture}
  \caption{ANTLR Tree walker generation}
  \label{fig:antlr_tree_walker}
\end{figure}

\subsection{LLVM}
\label{subsec:llvm}

LLVM\footnote{http://llvm.org} is a compiler framework designed to
support transparent, life-long program analysis and transformation for
arbitrary programs, by providing high-level information to compiler
transformations at compile-time, link-time, run-time, and in idle time
between runs \cite{LLVM:CGO04}.

Traditional compilers were tailored for only a few languages (with the
exception of GCC). However, all traditional compilers suffer from the
large inter-dependency of the basic blocks (Front-end, Optimizer,
Back-end). LLVM tries to solve this by providing an intermediate form
called the LLVM IR. A typical flow involving the basic blocks is
depicted in Figure \ref{fig:llvm_flow}.

\begin{figure}[hb!]
  \centering
   \begin{tikzpicture}[>=stealth]
    \tikzstyle{lang}=[rectangle,draw=black,thin,font=\tiny,inner
    sep=0pt, align=center,minimum width=2.7cm,minimum height=2.2em]

    \tikzstyle{txt}=[font=\tiny]

    \node[lang](llvm_opt){LLVM \\ Optimizer};
    \node[lang](ppc_back)[right=1 cm of llvm_opt]{LLVM \\ PowerPC Backend};
    \node[txt](ppc)[right=of ppc_back]{PowerPC};
    \node[lang](x86_back)[above of=ppc_back]{LLVM \\ x86 Backend};
    \node[txt](x86)[right=of x86_back]{x86};
    \node[lang](arm_back)[below of=ppc_back]{LLVM \\ ARM Backend};
    \node[txt](arm)[right=of arm_back]{ARM};

    \node[lang](gcc_front)[left=1 cm of llvm_opt]{llvm-gcc \\ Frontend};
    \node[txt](fortran)[left=of gcc_front]{Fortran};
    \node[lang](clang_front)[above of=gcc_front]{Clang C/C++/ObjC \\ Frontend};
    \node[txt](c)[left=of clang_front]{C};
    \node[lang](ghc_front)[below of=gcc_front]{GHC \\ Frontend};
    \node[txt](haskell)[left=of ghc_front]{Haskell};

    \draw[->] (clang_front.east) -- (llvm_opt.west);
    \draw[->] (gcc_front.east) -- (llvm_opt.west);
    \draw[->] (ghc_front.east) -- (llvm_opt.west);

    \draw[->] (llvm_opt.east) -- (x86_back.west);
    \draw[->] (llvm_opt.east) -- (ppc_back.west);
    \draw[->] (llvm_opt.east) -- (arm_back.west);

    \draw[->] (x86_back) -- (x86);
    \draw[->] (ppc_back) -- (ppc);
    \draw[->] (arm_back) -- (arm);

    \draw[->] (c) -- (clang_front);
    \draw[->] (fortran) -- (gcc_front);
    \draw[->] (haskell) -- (ghc_front);
  \end{tikzpicture}
  \caption{LLVM typical workflow \cite{llvm_general}}
  \label{fig:llvm_flow}
\end{figure}

Due to its modularity, LLVM has recently seen increased usage in a number
of independent fields:
\begin{itemize}
\item implementing a C/C++ compiler (Clang)
\item implementing C-to-HDL translation
\item implementing a Haskell compiler (GHC)
\item implementing Secure Virtual Architectures (SVA)
\item implementing dynamic translation
\item implementing OpenGL drivers (Mac OS X)
\item implementing OpenCL drivers (AMD)
\end{itemize}

\subsubsection{LLVM IR}

The LLVM IR is the intermediate representation language of the LLVM
project. On its own it is a first-class language with well defined
semantics \cite{llvm_general, llvm_master_thesis}. Variables are in
the SSA (Static Single Assignment) form meaning that they can be only
assigned once and they keep that value for their entire lifetime. All
the values residing in memory need to be loaded to a variable first
and stored back to memory if they wish to be saved. Instructions
operate solely upon variables. In this respect, the LLVM IR resembles
the assembly language of an infinitely many registers Load-Store based
RISC processor. The following snippet, taken from \cite{llvm_general},
demonstrates how two different functions implemented in C (one without
recursion and one with recursion) are compiled to LLVM IR:

\begin{lstlisting}[language=C]
unsigned add1(unsigned a, unsigned b) {
  return a+b;
}

unsigned add2(unsigned a, unsigned b) {
  if (a == 0) return b;
  return add2(a-1, b+1);
}
\end{lstlisting}

\begin{lstlisting}
define i32 @add1(i32 %a, i32 %b) {
entry:
  %tmp1 = add i32 %a, %b
  ret i32 %tmp1
}

define i32 @add2(i32 %a, i32 %b) {
entry:
  %tmp1 = icmp eq i32 %a, 0
  br i1 %tmp1, label %done, label %recurse

recurse:
  %tmp2 = sub i32 %a, 1
  %tmp3 = add i32 %b, 1
  %tmp4 = call i32 @add2(i32 %tmp2, i32 %tmp3)
  ret i32 %tmp4

done:
  ret i32 %b
}
\end{lstlisting}

A central concept while constructing LLVM IR is the
\emph{Module}~\cite{llvm_ir}, corresponding loosely to a source
file. A \emph{Module} consists of functions, global variables and
symbol table entries.

\subsection{GMP}
\label{subsec:gmp}

GMP\footnote{http://gmplib.org/} is a multi-precision library
supporting signed integers, rational numbers and floating-point
numbers with precision limited only the available memory. The library
provides a consistent calling interface across the different
supporting types. Main target applications of the library are
cryptography and computer algebra systems. The library itself aims to
be highly efficient by using different algorithms for different
operand sizes as well as extensive use of inline assembly and
additional processor instruction sets.

\subsection{GEZEL}
\label{subsec:gezel}

GEZEL is a cycle-accurate hardware description language (HDL) using the
Finite-State-Machine + Datapath (FSMD) model \cite{gezel}.

The basic element is a Signal Flow Graph. It groups operations that
are to be executed concurrently in the same clock cycle.
\begin{lstlisting}[language=GEZEL]
sfg increment {
  a = a + 1;
}
\end{lstlisting}
One or more of these SFGs are used to form a datapath which is the
main building block. It is the smallest GEZEL unit that can stand on
its own and be simulated \cite{gezel}. A datapath can be thought of as
a \emph{module} in Verilog or an \emph{entity} in VHDL. Here is a
full contained example of a counter in GEZEL:
\begin{lstlisting}[language=GEZEL]
dp counter(out value : ns(2)) {
   reg c : ns(2);
   always {
     value = c;
     c = c + 1;
     $display("Cycle ", $cycle, ": counter = ", value);
   }
}

system S {
  counter;
}
\end{lstlisting}

Figure \ref{fig:gezel_workflow} shows how the GEZEL language
can be used as an input to:
\begin{itemize}
\item fdlvhd - a generator that can generate synthesizeable VHDL or
  Verilog
\item fdlsim - a cycle accurate simulator used to verify and validate
  the design
\item gplatform - a co-simulation tool used for HW/SW co-design purposes
\end{itemize}

\begin{figure}[hb!]
  \centering
  \begin{tikzpicture}[>=stealth, font=\tiny]
    \tikzstyle{edge from parent}=[draw,->]

    \Tree[.\node[language](fdl){GEZEL language \\ *.fdl};
      [.\node[compiler](fdlvhd){Code Generator \\ fdlvhd};
        [.\node[language](vhdl){VHDL \\ Verilog};]
      ]
      [.\node[compiler](fdlsim){Simulator \\ fdlsim};
        [.\node[language](sim){Verification \\ Profiling};]
      ]
      [.\node[compiler](gplatform){CoSimualtor \\ gplatform};
        [.\node[language](cosim){Verification \\ Profiling};]
      ]
    ]
  \end{tikzpicture}
  \caption{GEZEL workflow \cite{gezel}}
  \label{fig:gezel_workflow}
\end{figure}

The co-simulation tool allows to cosimulate GEZEL designs with
instruction-set simulations \cite{gezel}. Supported processors are
ARM, AVR, 8051, MicroBlaze and PicoBlaze. The cosimulation tool allows
for designing a processor-coprocessor pair for a general purpose
processor and a custom dedicated coprocessor.

%%% Local Variables:
%%% TeX-PDF-mode: t
%%% TeX-master: "thesis"
%%% End:
