\chapter{HW - SW codesign project}
\label{cha:codesign_project}
\author{Boran Car \and Victor Statescu}

The goal of this project was to design a cryptographic system that
would support 1024-bit encryption RSA and ElGamal. The processor given
was an 8051-based. The choice is similar to what many smart-card
manufactures offer \cite{smartcard_crypto_coprocs,
  smartcard_crypto_coprocs2}. Basically, the 8051 is an industry
proven, well tested and widespread micro-controller. This it owes to
having been around for more than 20 years.

Since the processor was not fast or powerful enough, HW-SW co-design
was employed to offload some operations onto a separate coprocessor.

\section{System design}

The 8051 has 4 8-bit ports available for IO. Due to port sharing, not
all of these ports can be used in specific configurations (e.g. using
an external memory takes away the ports P0 and P2; using interrupts,
UART or external ROM takes away port P3). This was the reason why only
port P1 was chosen as a means of port IO signalization with the
coprocessor. Only 2 pins were used for the signalization so the rest
can still be used for other purposes.  

Shared memory was used as a means of communicating the data to the
coprocessor.  Addresses 0x600--0x801 are mapped to the coprocessor's
addresses 0x000--0x201. Addresses 0x000--0x600 are left for the main
processor.

The operations chosen for offloading to the coprocessor were the
Montgomery multiplication and Montgomery inversion for speed and
flexibility reasons.

The fastest the Montgomery multiplication could be made in SW was
$\unit[1.3]{s}$@$\unit[12]{MHz}$ which is not fast enough. HW
realization on the other hand took $\unit[2300]{cycles}$, which goes
roughly to $\unit[0.2]{ms}$@$\unit[12]{MHz}$.

For the inversion case it took $\unit[16]{s}$@$\unit[12]{MHz}$, while
in the HW realization $\unit[1.3]{mil. cycles}$, roughly
$\unit[100]{ms}$@$\unit[12]{MHz}$.

For the modular exponentiation case it was decided to keep it in SW
using the Montgomery multiplication of the coprocessor. Since the main
processor is 12 times slower some techniques have been applied to
eliminate the communication overhead.

The frequency of the system was lowered to $\unit[4]{MHz}$ to match
most of the manufacturers.

\section{Coprocessor design}

The coprocessor has:
\begin{itemize}
\item 6 1024-bit registers (u, v, p, R, S, Q)
\item 1 1024-bit datapath (adder and a shifter)
\item 10-bit datapaths and registers for loop and address counters
\end{itemize}

\noindent
The shared memory is used in the following way:
\begin{itemize}
\item 0x600--0x680 1024-bit number
\item 0x680--0x700 1024-bit number
\item 0x700--0x780 1024-bit number
\item 0x780--0x800 command queue (up to 128 commands)
\item 0x800 state signaling from the coprocessor
\end{itemize}

\noindent
The instructions are the following:
\begin{itemize}
\item Halt - stops the execution of the coprocessor and signals the
  done to the main processor
\item Init - initializes the coprocessor (just sets the modulus for now)
\item Montgomery multiplication - executes the Montgomery
  multiplication on the registers u and v and stores the result back
  to u
\item Montgomery squaring - executes the Montgomery squaring of the
  register u and stores the result back to u
\item Montgomery inversion - executes the Montgomery inversion of the
  register u and stores the result back to u (destroys the previous
  values of u and v)
\item Load u from the shared memory - loads the register u from the
  shared memory location 0x00 or 0x80 or 0x100
\item Load v from the shared memory - loads the register v from the
  shared memory location 0x00 or 0x80 or 0x100
\item Store result to shared memory - stores the result of the
  operation to memory location 0x00 or 0x80 or 0x100
\item Store quotient to memory (Montgomery only) - stores the quotient
  of the Montgomery operation to memory location 0x00 or 0x80 or 0x100
\end{itemize}

Special instructions were provided for loading from shared memory
because the coprocessor is 24 times faster with loading from memory
(12 clock cycles for 1 instruction cycle, 2 instruction cycles for
loading on the processor). This way, a third parameter can be stored
in the shared memory for faster fetching.

\subsection{Command queueing}

The idea is to allocate special storage in the shared memory for
the commands the coprocessor should execute. The main processor then
simply fills this memory with the commands the coprocessor needs to
execute. The main processor then starts the coprocessor which will
execute these instructions.

This is also useful if the main processor should be able to perform
other operations while the coprocessor executes the commands
provided. The method has also been tested with the main processor being
slower $144$ (additional $12\textrm{x}$ slowdown) times than the coprocessor
and it gave satisfying results (only $5\textrm{x}$ increase in cycle numbers for
RSA).

\subsection{Datapath}

The datapath design revolves around a 1024-bit adder and a shifter
in series. The shifter can shift one position to the left, one position
to the right or just pass the input.

The inputs to the adder are multiplexed between 0, 1, u, v, p, R, S,
power for the x operand. For the v operand, the multiplication step is
multiplexed instead of the power.

The result always gets assigned back to u. This was to allow chaining
of Montgomery multiplications when exponentiation is performed. This
way, no redundant copying is necessary from/to the shared memory. What
is also possible this way is to update the shared memory while the
coprocessor is executing (this helps the elimination of the
communication overhead). The technique was not tested within
this project.

\subsection{Modifications to the Montgomery product}

The Montgomery product computation algorithm has been modified to
include computing the quotient (as is called in
\cite{monpro_doubling}). This allows for doubling the bit-length
(\cite{monpro_doubling, classic_doubling}) of the crypto-coprocessor
in software. This technique was not tested within this project.

\section{Implementation}

\subsection{Software in C}

A library is provided exposing primitives which are executed on the
coprocessor:
\begin{itemize}
\item montpro - Montgomery product
\item montinv - Montgomery inversion
\item modexp - Modular exponentiation
\end{itemize}

\noindent
Also provided are the following software methods:
\begin{itemize}
\item add1024 - adding 1024-bit numbers
\item subtract1024 - subtracting 1024-bit numbers
\item multiply1024 - multiplies 1024-bit numbers to produce a 2048-bit
\item larger\_or\_equal - checks if the number is larger or equal than a number
\end{itemize}

These operations were left in software as they were already fast
enough there. Multiplication was left in case someone would want to
use the bit doubling method. These functions are documented in the
lib.h header file.

\subsection{Hardware in VHDL}

During the co-design phase, separate datapaths were made for the adder
and the bit shifter.  This was to allow a different implementation in
VHDL as the GEZEL-to-VHDL tool (fdlvhd) generates one .vhd file per
datapath. The target devices usually have sophisticated CLA logic
implemented so that efficient adders in terms of space and speed can
be generated.

The same reasoning goes for ASIC design. Usually the libraries
provided from the foundries will include efficient designs of adders
which can be combined to produce the required adder. Even if they
don't, having a separate block allows the designer to concentrate more
on where optimization really counts (critical data path, major effect
on area).

In our specific case, the GEZEL-to-VHDL tool generated 2 1024-bit
adders, one with carry-in and one without it. Manual intervention was
required to change it to a single 1024-bit adder with carry-in. Later
on 2 513-bit adders were used to reduce the slice count even further.
Putting 2 smaller adders helped the synthesis tool spread out and
perform better interconnect.
Custom add\_sub.customxx.vhd are provided in the vhdl subfolder.

\section{Results}

The RSA encryption and decryption with provided exponents (9-bit for encryption and 1024-bit for decryption) take around $\unit[5]{mil. cycles}$ to complete in total. On the $\unit[4]{MHz}$ clock frequency this amounts to around $\unit[1.25]{s}$ (average for encryption/decryption $\unit[600]{ms}$).

The ElGamal encryption and decryption with provided exponents (128-bit for encryption and decryption) take around $\unit[4.5]{mil. cycles}$ to complete in total. On the $\unit[4]{MHz}$ clock frequency this amounts to around $\unit[1.125]{s}$ (average for encryption/decryption $\unit[560]{ms}$).

These times put the design along with most of the systems presented in \cite{smartcard_crypto_coprocs, smartcard_crypto_coprocs2}.

Eventually the implemented design did not meet the area constraint. It
used $101\%$ of the device (Table \ref{tab:results}). The maximal
frequency is $\unit[20.449]{MHz}$ which is well above the required
$\unit[4]{MHz}$.

\begin{table}[hbt!]
  \centering
  \begin{tabular}[hbt!]{l|r|r|r}
                 & Used  & Available & \% \\
    \hline
    Slices       & $13951$ &     $13696$ & $101\%$ \\
    Flip-flops   &  $6267$ &     $27392$ &  $22\%$ \\
    4-input LUTs & $26033$ &     $27392$ &  $95\%$
  \end{tabular}
  \caption{Implementation results}
  \label{tab:results}
\end{table}

Following is an excerpt from the synthesis tools report (available as the file sys.
syr):

\begin{verbatim}
# Adders/Subtractors                                   : 5
 10-bit adder                                          : 2
 11-bit subtractor                                     : 1
 513-bit adder carry in                                : 2
# Registers                                            : 6251
 Flip-Flops                                            : 6251
# Comparators                                          : 1
 1024-bit comparator equal                             : 1
# Multiplexers                                         : 1
 1026-bit 4-to-1 multiplexer                           : 1
\end{verbatim}

%%% Local Variables: 
%%% TeX-PDF-mode: t
%%% TeX-master: "thesis"
%%% End: 
