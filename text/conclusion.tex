\chapter{Conclusions and Future Work}

\section{Conclusions}

Previous compiler frameworks for Zero Knowledge Proofs of Knowledge
(ZKPK) targeted general purpose processors and had no HW-SW co-design
exploration capabilities. We have shown that HW-SW co-design is
possible by extracting the Data Flow Graph (DFG) and have proven the
PIL language expressive, yet simple enough for easy DFG extraction. By
implementing the two extremes of the co-design spectrum, we have
opened the path for automated exploration of the spectrum in the
``middle''. The changes we have introduced do not interfere with the
standard CACE ZKC work-flow as we allow previous CACE examples to work
with our framework as well. We have merely added extensions to PIL
that could later prove useful outside of the CACE ZKC work-flow.

The extensions to the PIL Language have allowed more complex,
multiparty protocols as well as parameter specification inside of PIL
files. Protocols can be implemented simply by writing each of the
steps of the protocol flow in the PIL language. The transformation is
also reversible thanks to the CACE ZKC \LaTeX{} output. The advantages
of a domain specific language have clearly been shown throughout this
thesis.

Our extensions to the CACE ZKC support library have allowed
communication over serial ports and thus multiple combinations of
general purpose computer and embedded device interconnectivity can be
made. Serial port communication is easier to implement and it can also
serve for cross-checking and cross-validating of custom
implementations.

Finally, our framework is easily extendable to support new points in
the HW-SW co-design spectrum as we have clearly shown with our use
case. The custom system targeted by the use case is not unlike the
commodity hardware available for these purposes and the protocol
chosen, albeit simple, is still of practical value.

\section{Future Work}

We believe that this work should eventually give rise to an automated
HW-SW co-design framework where the user will be offered fine-grained
choice of trade-offs. Automated HW-SW balancing could be made that
allows trading speed for smaller device size and less power
consumption.

We also believe that LLVM should be extended with the modular residue
group types and this work should be submitted upstream to the
development tree of LLVM. The gains were already discussed in Section
\ref{sec:pil_frontend} and mostly deal with verifiability and
security. The process requires some standardization before changes are
applied and as this takes time, it was not pursued within the course
of this thesis.

Eventually, we think that our framework should support automatic
parameter generation as leaving these choices to the end user might be
problematic from a usage perspective as well as a security
perspective. The user simply cannot and should not know all the needed
constraints of the protocol and the only constraints he should be
concerned about are implementation constraints. The framework
developed within the course of this thesis already allows for compile
time/constant expression and these expressions can be used as a
starting point for defining a generator macro/function that should
automatically generate the required parameters.

Finally, while extending the CACE ZKC, it was noticed that each of the
tools from the CACE Project have their own lower-level language. In
the long run, this only makes it difficult to contribute new
features/extend, optimize and target new platforms. We find it an
interesting possibility for the LLVM IR to become a common language
for the entire CACE Project.

%%% Local Variables: 
%%% TeX-PDF-mode: t
%%% TeX-master: "thesis"
%%% End: 
