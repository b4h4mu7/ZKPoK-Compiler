\chapter{Conclusion}

\section{Future work}

LLVM should be extended with the modular residue group types and this
work should be submitted upstream. The gains were already discussed in
section \ref{sec:pil_frontend}. This will require some standardization
before changes are applied and will take some time. This is one of the
main reasons it was not done within the course of this thesis.

While extending the CACE Project Zero Knowledge Compiler, it was
noticed that each of the tools have their own lower-level language. In
the long run, this only makes it difficult to contribute new
features/extend, optimize and target new platforms. It is possible for
LLVM to become a common language for all the tools from the CACE
Project. This not only makes it portable to a wider range of platforms
but it also moves the burden of optimizations to an already proven
framework.

The framework developed during the course of this thesis could and
should serve as a starting point for HW-SW codesign automation tools.
Due to lack of time, and lack of support in the simulators, limited
exploration was done in this context, targeting only GEZEL code. There
is no reason not to include C-GEZEL code co-generation and employ an
8051. Such a micro-controller is usually found on modern-day
smart-cards.

%%% Local Variables: 
%%% TeX-PDF-mode: t
%%% TeX-master: "thesis"
%%% End: 
