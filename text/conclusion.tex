\chapter{Conclusions and Future Work}

\section{Conclusions}

Previous compiler frameworks for Zero Knowledge Proofs of Knowledge
targeted general purpose processors. It has been shown that the PIL
language is expressive, yet abstract enough for Data Flow Graph (DFG)
extraction. A direct hardware realization has thus been made possible
while also allowing the implementation to target general purpose
processors. This opens a path for Control Flow Graph (CFG) generation
such that hardware software co-design can be done in an automated way
with a high-level control from the user. The changes introduced do not
interfere with the standard CACE work-flow as it is still possible to
define a PSL and generate a PIL from it. The advantage is that
multiple architecture are now supported.

The extensions to the CACE Project Zero Knowledge Compiler support
library have allowed communication over serial ports and have thus
allowed multiple combinations of general purpose computer and embedded
device interconnectivity. Serial port communication is easier to
implement and this can also serve for cross-checking and
cross-validating of custom implementations.

The extensions to the PIL Language have allowed a complex protocols
such as the Direct Anonymous Attestation (DAA) Join protocol to be
implemented easily by just translating the protocol flow specification
to PIL. The transformation is also reversible thanks to the CACE
Project Zero Knowledge Compiler \LaTeX{} output. The advantages of a
domain specific language have clearly been shown.

\section{Future Work}

LLVM should be extended with the modular residue group types and this
work should be submitted upstream to the development tree of LLVM. The
gains were already discussed in Section \ref{sec:pil_frontend}. This
will require some standardization before changes are applied and will
take some time. This is one of the main reasons it was not done within
the course of this thesis.

While extending the CACE Project Zero Knowledge Compiler, it was
noticed that each of the tools from the CACE Project have their own
lower-level language. In the long run, this only makes it difficult to
contribute new features/extend, optimize and target new platforms. It
is possible for LLVM to become a common language for all the tools
from the CACE Project. This not only makes it portable to a wider
range of platforms but it also moves the burden of optimizations to an
already proven framework.

The CACE Project Zero Knowledge Compiler does not support automatic
generation of parameters. This is something an extension of this
framework should also allow. Leaving these choices to the end user
might be problematic both from a usage perspective as well as security
perspective. The user simply cannot and should not know all the needed
constraints. The only constraints the user should set are the bit
sizes. The framework developed within the course of this thesis
already allows for compile time/constant expression. This expressions
can be used as a starting point for defining a generator
macro/function that should automatically generate the required
parameters.

The framework developed during the course of this thesis could and
should serve as a starting point for HW-SW codesign automation tools.
Due to lack of time, and lack of support in the simulators, limited
exploration was done in this context, targeting only GEZEL code. There
is no reason not to include C-GEZEL code co-generation and employ an
8051. Such a micro-controller is usually found on modern-day
smart-cards. This would require writing an 8051 back-end.

There is no reason not to go beyond writing LLVM back-ends. A pseudo
Virtual Machine can be realized by function calls in C. Such a Virtual
Machine is then easily deployable to even more architectures as C is
an industry standard compiler.

The generated hardware uses a primitive CFG and implements the entire
DFG. A DFG uniquely defines an algorithm, while a CFG only defines the
execution. Automated CFG-DFG balancing (controlled via a parameter)
could be made that allows trading speed for smaller device size and
less power consumption.

%%% Local Variables: 
%%% TeX-PDF-mode: t
%%% TeX-master: "thesis"
%%% End: 
