
%\label{antlr}

ANTLR\footnote{http://www.antlr.org} (ANother Tool For Language
Recognition) is a parser generator tool that generates LL(*)
parsers. The tool accepts grammar definitions as input files and
produces output in the target language which can be chosen among C,
Java or Python.

The input to ANTLR is a context-free grammar that must be of the LL
form. This means that there should be no left recursion or ambiguities
when encountering the first element on the left. Left factoring is
usually used to solve this, but this can sometimes lead to rules which
have counter intuitive representation. The grammar can be augmented with
syntactic and semantic predicates to cope with this~\cite{ANTLR,ANTLR2}.

ANTLR can generate both lexers and parsers. The two grammars can be
combined in a single file.  In such cases, parser rules are written in
lowercase, while the lexer rules are written in uppercase. Upon
generation, two separate entities will be created, a parser source and
a lexer source in the target language (as illustrated in Figure~\ref{fig:antlr_lexer_parser_walker}).

\begin{figure}[hb!]
  \centering
  \subfloat{
  \begin{tikzpicture}[>=stealth, font=\tiny]
    \tikzstyle{edge from parent}=[draw,->]

    \Tree[.\node[language](parser_g){Grammar \\ *.g};
      [.\node[compiler](antlr){ANTLR};
        [.\node[compiler](lexer){Lexer};]
        [.\node[compiler](parser){Parser};]
      ]
    ]
  \end{tikzpicture}
  } \qquad
  \subfloat{
  \begin{tikzpicture}[>=stealth, font=\tiny]
    \tikzstyle{edge from parent}=[draw,->]

    \Tree[.\node[language](parser_g){Tree Grammar \\ *.g};
      [.\node[compiler](antlr){ANTLR};
        [.\node[compiler](lexer){Tree walker};]
      ]
    ]
  \end{tikzpicture}
  }
  \caption{ANTLR Lexer, parser and tree walker flow}
  \label{fig:antlr_lexer_parser_walker}
\end{figure}

The output of the parser generated by ANTLR is a Parse Tree. ANTLR
also allows specifying a tree transformation to apply to this Parse
Tree. This can be used to automatically generate an Abstract Syntax
Tree (AST).

ANTLR can also generate a tree parser/walker that visit each node of
the AST and applies a certain operation or produces a certain
output. The generation of such a walker is illustrated in Figure~\ref{fig:antlr_lexer_parser_walker}.
