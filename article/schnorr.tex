
The Schnorr's Identification Protocol is a simple
$\Sigma$-protocol. Here we demonstrate how our extensions do not
disrupt the CACE Project ZKC flow but merely complement it. An input
PSL file is given to the CACE Project ZKC and a corresponding PIL file
is received. This PIL file is processed using our framework to
generate both a GEZEL and a GMP target. These are then cross-validated
with the C code that the CACE Project ZKC generates. This was made
possible thanks to the implemented terminal port communication
library.

\subsection{A custom co-processor system}

We took a step further into demonstrating hardware-software co-design
using our framework and have wrote a back-end for an existing
co-processor system. The main processor of this system is an 8051 as
is commonly found on most
smart-cards~\cite{smartcard_crypto_coprocs2}. The 8051 is a very
limited processor and a custom co-processor is needed if cryptography
applications are to be targeted. This leads to a trade-off in terms of
execution time and size as the more operations are implemented in the
co-processor the execution time is lower but the size required is
larger. The project deemed the Montgomery multiplication the most
critical and has implemented it as the only operation on the
co-processor. Since the main processor is slower than the
co-processor, it was deemed necessary to add command queuing so that
the co-processor can execute while the main processor is performing
some other task.

The back-end's job was merely to reuse the existing libraries provided
by the co-processor designers but we still believe it was sufficient
to show the feasibility of the approach of this framework. The
execution time was around $2,000,000$ cycles which on a
$\unit[4]{MHz}$ system (common frequency as mentioned in
\cite{smartcard_crypto_coprocs2}) is roughly $\unit[0.5]{s}$.

%%% Local Variables: 
%%% TeX-PDF-mode: t
%%% TeX-master: "main"
%%% End:
