%-	Importance of Zero-Knowledge Proofs
%-	Some Compilers have been designed to ease implementation of ZKPK
%-	Existing compilers are not suitable embedded devices
%
%-	We extend an existing compiler (CACE) to allow for ZKPK implementations on
%in HW specification languages (mention some) and on embedded devices (mentioned some devices).
%-	It also allows for HW-SW co-design and for verification and validation of the resulting implementation.

\begin{abstract}

Zero-knowledge proofs of knowledge (ZKPK) are the key building block of numerous cryptographic schemes, such as anonymous authentication or anonymous e-cash. Implementing ZKPK has proved to be error-prone, which hinders the widespread deployment of such schemes. To ease implementation work, several cryptographically aware compilers and libraries have been proposed to provide ZKPK implementations in high-level languages such as C, C++ and Java. However, some cryptographic schemes are expected to be implemented in resource-constrained devices, such as RFID tags (anonymous authentication) or smart cards (anonymous e-cash). In such situations, a ZKPK compiler that targets hardware description languages and hardware-software co-design is convenient.

We propose a custom ZKPK compiler framework that extends the CACE ZKPK compiler (ESORICS 2010) to enable hardware-software co-design. Concretely, our ZKPK compiler takes as input platform-independent implementations generated by the CACE ZKPK compiler and transforms them into implementations in LLVM IR, the representation language of the LLVM compiler framework. From LLVM IR, one can target multiple platforms. We provide a back-end for GEZEL, a hardware description language which also enables hardware-software co-design. We also provide a back-end for C+GMP, thus covering both ends of the co-design spectrum.



%Zero-Knowledge Proofs of Knowledge (ZKPK) are cryptographic primitives
%that allow a player to prove the veracity of a statement without revealing
%anything beyond the validity of the statement. Due to its strong privacy
%properties ZKPK have become essential building blocks in state-of-the-art
%privacy preserving applications, the Direct Anonymous
%Attestation (DAA) employed by Trusted Platform Modules (TPM) being the most
%prominent example.
%
%The underlying complexity of ZKPK has prompted
%the appearance of numerous cryptographically aware compilers, typically targeting
%(one of many) high level programming languages such as C, C++, or Java.
%While these tools are suitable for software oriented platforms, so
%far no line of research has focused on the hardware perspective and, more particularly,
%on the hardware/software co-design boundaries. We believe this
%to be of importance as privacy preserving applications are
%expected to run on small embedded devices such as RFID tags or smart cards.
%
%In this work we present a custom compiler framework for ZKPK, based on the CACE ZKPK Compiler, that allows
%designers to explore the full hardware/software co-design spectrum. Our framework
%generates platform independent code compatible with the LLVM compiler suite, thus
%allowing designers to choose from multiple software target languages and platforms. At the same
%time, we provide a back end for the Gezel hardware description language, offering
%a co-simulation environment ideal for co-design exploration.


  %Zero-Knowledge Proofs of Knowledge allow one to prove possession or
%  knowing of a certificate without actually revealing the secret. This
%  involves using a hard problem as a basis for the protocol. Goals of
%  current Zero-Knowledge Proofs of Knowledge compiler frameworks are
%  to make this process easy for those outside the field of
%  cryptography while at the same time making it easy for those inside
%  the field to develop such protocols. We acknowledge that such
%  frameworks have made it easy to develop the protocols but we note
%  that they have not made it easy or general to those implementing the
%  protocols on the end devices. Such frameworks either target C/C++
%  code generation or an interpreted language. They also use high-level
%  multi-precision arithmetic libraries which are too resource hungry
%  for small embedded devices. These limitations as well as hidden
%  pitfalls of C/C++ have motivated us to extend one of those
%  frameworks and build our own framework from it. This framework is
%  targeted specifically at small embedded devices while not losing any
%  generality of the framework it extends. We hope that this will ease
%  the work for those outside the field and allow for a more widespread
%  usage of these protocols.

\begin{keywords}
zero-knowledge proof of knowledge, cryptographically aware compilers, hardware-software co-design
\end{keywords}

\end{abstract}

%%% Local Variables:
%%% TeX-PDF-mode: t
%%% TeX-master: "main"
%%% End:
