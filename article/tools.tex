%-	ANTLR
%-	LLVM
%-	GEZEL



\subsection{LLVM: A Compiler Framework}
\label{llvm}

LLVM\footnote{http://llvm.org} is a compiler framework designed to
support transparent, life-long program analysis and transformation for
arbitrary programs, by providing high-level information to compiler
transformations at compile-time, link-time, run-time, and in idle time
between runs \cite{LLVM:CGO04}.

Traditional compilers were tailored for only a few languages (with the
exception of GCC). However, all traditional compilers suffer from the
large inter-dependency of the basic blocks (Front-end, Optimizer,
Back-end). LLVM tries to solve this by providing an intermediate form
called the LLVM IR. A typical flow involving the basic blocks is
depicted in Figure \ref{fig:llvm_flow}.

\begin{figure}[hb!]
  \centering
   \begin{tikzpicture}[>=stealth]
    \tikzstyle{lang}=[rectangle,draw=black,thin,font=\tiny,inner
    sep=0pt, align=center,minimum width=2cm,minimum height=2.2em]

    \tikzstyle{txt}=[font=\tiny]

    \node[lang](llvm_opt){LLVM \\ Optimizer};
    \node[lang](ppc_back)[right=1 cm of llvm_opt]{LLVM \\ PowerPC Backend};
    \node[txt](ppc)[right=of ppc_back]{PowerPC};
    \node[lang](x86_back)[above of=ppc_back]{LLVM \\ x86 Backend};
    \node[txt](x86)[right=of x86_back]{x86};
    \node[lang](arm_back)[below of=ppc_back]{LLVM \\ ARM Backend};
    \node[txt](arm)[right=of arm_back]{ARM};

    \node[lang](gcc_front)[left=1 cm of llvm_opt]{llvm-gcc \\ Frontend};
    \node[txt](fortran)[left=of gcc_front]{Fortran};
    \node[lang](clang_front)[above of=gcc_front]{Clang C/C++/ObjC \\ Frontend};
    \node[txt](c)[left=of clang_front]{C};
    \node[lang](ghc_front)[below of=gcc_front]{GHC \\ Frontend};
    \node[txt](haskell)[left=of ghc_front]{Haskell};

    \draw[->] (clang_front.east) -- (llvm_opt.west);
    \draw[->] (gcc_front.east) -- (llvm_opt.west);
    \draw[->] (ghc_front.east) -- (llvm_opt.west);

    \draw[->] (llvm_opt.east) -- (x86_back.west);
    \draw[->] (llvm_opt.east) -- (ppc_back.west);
    \draw[->] (llvm_opt.east) -- (arm_back.west);

    \draw[->] (x86_back) -- (x86);
    \draw[->] (ppc_back) -- (ppc);
    \draw[->] (arm_back) -- (arm);

    \draw[->] (c) -- (clang_front);
    \draw[->] (fortran) -- (gcc_front);
    \draw[->] (haskell) -- (ghc_front);
  \end{tikzpicture}
  \caption{LLVM typical workflow \cite{llvm_general}}
  \label{fig:llvm_flow}
\end{figure}

\subsubsection{LLVM IR.}

The LLVM IR is the intermediate representation language of the LLVM
project. On its own it is a first-class language with well defined
semantics \cite{llvm_general,llvmmasterthesis}. Variables are in the SSA (Static Single
Assignment) form meaning that they can only be assigned once and they
keep that value for their entire lifetime. All the values residing in
memory need to be loaded to a variable first and stored back to memory
if they wish to be saved. Instructions operate solely upon
variables. In this respect, the LLVM IR resembles the assembly
language of an infinitely many registers Load-Store based RISC
processor.

The central concept in constructing LLVM IR is the Module. Each module
consists of functions, global variables and symbol table entries.
Modules can be combined using the LLVM linker \cite{llvm_ir}.

\subsection{GEZEL: A Hardware Description Language}
\label{gezel}

GEZEL\footnote{http://rijndael.ece.vt.edu/gezel2/} is a cycle-accurate hardware description language (HDL) using the
Finite-State-Machine + Datapath (FSMD) model. The basic element is a Signal Flow Graph (SFG). It groups operations that
are to be executed concurrently in the same clock cycle. One or more of these SFGs are used to form a datapath, which is the
main building block. A datapath is the smallest GEZEL unit that can stand on
its own and be simulated. Datapaths can be thought of as
\emph{modules} in Verilog or \emph{entities} in VHDL.

Figure \ref{fig:gezel_workflow} shows how the GEZEL language
can be used as an input to the following tools: \emph{fdlvhd}, a generator that can output synthesizeable VHDL or Verilog code; \emph{fdlsim}, a cycle accurate simulator used to verify and validate the design; and \emph{gplatform}, a co-simulation tool used for HW/SW co-design purposes.

\begin{figure}[hb!]
  \centering
  \begin{tikzpicture}[>=stealth, font=\tiny]
    \tikzstyle{edge from parent}=[draw,->]

    \Tree[.\node[language](fdl){GEZEL language \\ *.fdl};
      [.\node[compiler](fdlvhd){Code Generator \\ fdlvhd};
        [.\node[language](vhdl){VHDL \\ Verilog};]
      ]
      [.\node[compiler](fdlsim){Simulator \\ fdlsim};
        [.\node[language](sim){Verification \\ Profiling};]
      ]
      [.\node[compiler](gplatform){CoSimulator \\ gplatform};
        [.\node[language](cosim){Verification \\ Profiling};]
      ]
    ]
  \end{tikzpicture}
  \caption{GEZEL workflow}
  \label{fig:gezel_workflow}
\end{figure}

The co-simulation tool allows to cosimulate GEZEL designs with
instruction-set simulations. Supported processors include
ARM, AVR, 8051, MicroBlaze and PicoBlaze. The cosimulation tool allows
for designing a processor-coprocessor pair for a general purpose
processor and a custom dedicated coprocessor~\cite{Schaumont}.

%%% Local Variables:
%%% TeX-PDF-mode: t
%%% TeX-master: "main"
%%% End:
